\label{implementation} We will first give a general overview of the approach we used, for more details see section ref{sec:details}

We have implemented tail recursion elimination, as earlier mentioned, this was done by introducing a new stage, after type checking, that transforms the AST. Changing tail recursive functions into functions with while loops.

Such a transformation goes through the following steps:

\begin{enumerate}
    \item Detect tail recursive functions, and for each:
    \item Introduce variables to hold arguments, and a special status variable for the main loop to loop on.
    \item Encase the method body in a while loop that loops on the status variable.
    \item Add variable initialization in the beginning of the body of the while loop.
    \item Rewrite all expressions in the while loop body to use the argument variables we introduced instead of arguments. 
    \item Rewrite the last expression of the while loop, removing recursive tail calls and replacing them with assigments to the argument variables we introduced and setting the status flag to loop.
    \item On all branches of the last expression that did not invoke a recursive call, add an assigment to the status variable to exit the loop.
\end{enumerate}

In section \ref{examples} we have a tail call recursion example with a class called OverflowPlease. When running this example without our optimization, the code results in a stackoverflow.

The code we can see bellow is the a pritty printed AST of the previous program after tail recursion elimination. We can see that the methodcall now have been replaced by a while loop. We can also observe the added variables that have been intoduced to be used instead of the methods arguments. Note that these have identification values starting with '\_'. This is because '\_' is not allowed as the starting char of an identifier making it impossible to create a duplication of an identifier since the code at this stage has already gone through the name analysis phase. 
\begin{lstlisting}
class OverflowPlease {
  def makeZero(i: Int): Int = {
    var _i : Int = i;
    var _LOOP : Boolean = true;
    var _RES : Int = 0;
    while (_LOOP) {
      _LOOP = false;
      if ((_i) == (0)) {
        _RES = _i
      }
      else {
        {
          _i = (_i) + (1);
          _LOOP = true
        }
      } /*end if((_i) == (0))*/
    } /* end while(_LOOP)*/;
    _RES
  }
}
object Main extends App {
  println(new OverflowPlease.makeZero(5)) 
}
\end{lstlisting}



\subsection{Terminalogy and Background}
An instruction in "the tail position" means that it is the last thing to be executed before the method exits.

A tail-recursion is a method with a call to itself in the tail position.

By tail call co-recursion we mean 2 or more methods that have calls to each other in the tail position.

By AST-transform, we mean a function that takes an AST and rewrites it into an other AST. In our case we want our AST transformation to preserve the effect of executing the AST.

\subsection{Implementation Details}
\label{sec:details}
We followed the bulletpoints stated in Section \ref{implementation}. We found that the implementation was quite straightforward and did not 
come across any major obsticles. However the design o the compiler made the optimization a bit gruesome. This because there is no ficility to easialy create copies or modified copies of AST nodes,
every time we want to rewrite a node we have to instantiate a new trie, copy the values and assign the symbols, types and possitions manually. It would have been nice to if we implemented a way to copy and update ASTs, especially if more optimizations where to be implemented.

When planning our implementation we forgot that while loops in punkt0 always have tipe Unit, So the while loop could not be placed in the retExpr of the function. We solved this by adding another variable \texttt{_RES} used to hold the output of the final recursive call, and return the value in the retExpr of the function.
