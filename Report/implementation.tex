We will first give a general overview of the apreach we inted to use, for more details see Section ref{sec:details}

We inted to implement tail recursion elemination as mentioned by introducing a new stage, after type checking, that transforms the AST. Changing tail recursive functions into functions with while loops.

such a transformatio would have to perform the following steps

\begin{enumerate}
    \item Detect tail recursive functions, and for each:
    \item Introduce variables to hold arguments, and a special status variable for the main loop to loop on
    \item Encase the method body in a while loop that loops on the status variable
    \item Add variable initializiton in the begining of the body of the while loop.
    \item Rewrite all expresions in the while loop body to use the argument variables we introduced insted of argumets 
    \item Rewrite the last expresion of the while loop removing recursive tail calls replacing them with assigments to the argument variables we introduced and seting the status flag to loop
    \item on all branches of the last expresion that did not invoce a recursive call, add a assigment to the status variable to exit the loop.
\end{enumerate}




\subsection{Terminalogy and Background}
An instrcution in "in the tail possition" means that it is the last thing to be executed before the method exits.

A tail-recursion is a method with a call to itself in the tail position

By tail call co-recursion we mean 2 or more methods that have calls to each other in the tail possition.


By AST-transform, we mean a function that takes an AST ard rewrites it into an other AST. In our case we want our AST transformations to perserve the effect of executing the AST.


\subsection{Implementation Details}
\label{sec:details}
Describe all non-obvious tricks you used. Tell us what you thought was hard and
why. If it took you time to figure out the solution to a problem, it probably
means it wasn't easy and you should definitely describe the solution in detail
here. If you used what you think is a cool algorithm for some problem, tell us.
Do not however spend time describing trivial things (we know what a tree traversal
is, for instance).

After reading this section, we should be convinced that you knew what you were
doing when you wrote your extension, and that you put some extra consideration
for the harder parts.
