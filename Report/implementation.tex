\label{implementation}
We will first give a general overview of the approach we intend to use, for more details see section ref{sec:details}

We intend to implement tail recursion elimination, as earlier mentioned, this will be done by introducing a new stage, after type checking, that transforms the AST. Changing tail recursive functions into functions with while loops.

Such a transformation would have to perform the following steps:

\begin{enumerate}
    \item Detect tail recursive functions, and for each:
    \item Introduce variables to hold arguments, and a special status variable for the main loop to loop on.
    \item Encase the method body in a while loop that loops on the status variable.
    \item Add variable initialization in the beginning of the body of the while loop.
    \item Rewrite all expressions in the while loop body to use the argument variables we introduced instead of arguments. 
    \item Rewrite the last expression of the while loop, removing recursive tail calls and replacing them with assigments to the argument variables we introduced and setting the status flag to loop.
    \item On all branches of the last expression that did not invoke a recursive call, add a assigment to the status variable to exit the loop.
\end{enumerate}




\subsection{Terminalogy and Background}
An instruction in "the tail position" means that it is the last thing to be executed before the method exits.

A tail-recursion is a method with a call to itself in the tail position.

By tail call co-recursion we mean 2 or more methods that have calls to each other in the tail position.

By AST-transform, we mean a function that takes an AST and rewrites it into an other AST. In our case we want our AST transformation to preserve the effect of executing the AST.

\subsection{Implementation Details}
\label{sec:details}
We followed the bulletpoints stated in Section \ref{implementation}. We found that the implementation was quite straightforward and did not 
come across any major obsticles. Though the previous implementation of the compiler made the optimization a bit gruesome. This because, when creating
new variables for example, we always had to assign the symbols and types manually. It would have been nice to have som automatization for this, 
especially if more optimizations where to be implemented.

Describe all non-obvious tricks you used. Tell us what you thought was hard and
why. If it took you time to figure out the solution to a problem, it probably
means it wasn't easy and you should definitely describe the solution in detail
here. If you used what you think is a cool algorithm for some problem, tell us.
Do not however spend time describing trivial things (we know what a tree traversal
is, for instance).

After reading this section, we should be convinced that you knew what you were
doing when you wrote your extension, and that you put some extra consideration
for the harder parts.
